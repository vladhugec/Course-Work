%%%%%%%%%%%%%%%%%%%%%%%
% Comp 160, Fall 2019
% Homework 1
% Author: Vladimir Hugec
%%%%%%%%%%%%%%%%%%%%%%%

% This portion of the LaTeX document are configuration 
% You can see it as all the #includes in C++
\documentclass[12pt]{article}

\usepackage{epsfig}
\usepackage{amsmath}
\usepackage{amsthm}
\usepackage{listings}
\usepackage{graphicx}
\usepackage{coffee4}

\newtheorem{lemma}{Lemma}
\newtheorem{theorem}{Theorem}

\usepackage{titlesec}
\titleformat{\section}
{\normalfont\Large\bfseries}{Question~\thesection:}{1em}{}

\newlength{\toppush}
\setlength{\toppush}{2\headheight}
\addtolength{\toppush}{\headsep}

\def\subjnum{Comp 160}
\def\subjname{Introduction to Algorithms}

\def\doheading#1#2#3{\vfill\eject\vspace*{-\toppush}%
  \vbox{\hbox to\textwidth{{\bf} \subjnum: \subjname \hfil Vladimir Hugec}%
    \hbox to\textwidth{{\bf} Tufts University, Fall 2019 \hfil#3\strut}%
    \hrule}}


\newcommand{\htitle}[1]{\vspace*{1.25ex plus 1ex minus 0ex}%
\begin{center}
{\large\bf #1}
\end{center}} 


%%%%%%%%%%%%%%%%%%%%%%%%%%%%%%%%%%%%%%%%%%%%%%%%%%%%%%%%%%%%%%%%%%%
% BEGIN DOCUMENT
%%%%%%%%%%%%%%%%%%%%%%%%%%%%%%%%%%%%%%%%%%%%%%%%%%%%%%%%%%%%%%%%%%%
\begin{document}
\doheading{2}{title}{Homework 1}

\section{$f(n) = 3n^2 + 10n + 729$}
\subsection{Prove that $f(n) = O(n^2)$}

\begin{proof}

To prove $f(n) = O(n^2)$ we must show that there exists a value, $n_{0}$, and a constant, $c$, such that:
\begin{center}
$ 3n^2 + 10n + 729 \leq cn^2 $,  $\forall n > n_{0}$
\end{center}
If we can show that such an $n_{0}$ and $c$ exist, that is the definition of Big O Notation.
Let $n_{0} = 27$ and $c = 15$. Since $n > 0$ we can upper bound $10n$ by $10n^2$:
\begin{center}
$ 3n^2 + 10n + 729 < 3n^2 +10n^2 + 729 = 13n^2 + 729$ 
\end{center}
Also since $n > n_{0} = 27$ this gives us $729 < n^2$:
\begin{center}
$13n^2 + 729 < 15n^2$
\end{center}
So, for $n > 27$, $f(n)$ was upper bounded by a constant multiplied by $n^2$. So by the definition of Big O Notation, the statement is shown.


(NOTE: Methodology for this proof comes from the ``How to write excellent proofs'' handout)
\end{proof}
\pagebreak

\subsection{Prove that $f(n) = O(n^3)$}

\begin{proof}
To prove $f(n) = O(n^2)$ we must show that there exists a value, $n_{0}$, and a constant, $c$, such that:
\begin{center}
$ 3n^2 + 10n + 729 \leq cn^3 $,  $\forall n > n_{0}$
\end{center}
If we can show that such an $n_{0}$ and $c$ exist, that is the definition of Big O Notation.
Let $n_{0} = 9$ and $c = 15$. Since $n > 0$ we can upper bound $10n$ by $10n^3$:
\begin{center}
$ 3n^2 + 10n + 729 < 3n^3 +10n^3 + 729 = 13n^3 + 729$ 
\end{center}
Also since $n > n_{0} = 9$ this gives us $729 < n^3$:
\begin{center}
$13n^2 + 729 < 15n^3$
\end{center}
So, for $n > 9$, $f(n)$ was upper bounded by a constant multiplied by $n^2$. So by the definition of Big O Notation, the statement is shown.


(NOTE: As with the previous Proof, the methodology for this one also comes from the ``How to write excellent proofs'' handout)

\end{proof}
\pagebreak
\subsection{Prove that $f(n) = \Omega(n)$}

\begin{proof}
To prove $f(n) = \Omega(n)$ we must show that there exists a \textbf{positive} value, $n_{0}$, and a positive constant, $c$, such that:
\begin{center}
$ 3n^2 + 10n + 729 \geq cn $,  $\forall n > n_{0}$
\end{center}
If we can show that such an $n_{0}$ and $c$ exist, that is the definition of $\Omega$ Notation.

Let $n_{0} = 0$ and $c = 1$. We know:
\begin{center}
$10n \geq n$

$10n + 729 \geq n$

$3n^2 + 10n + 729 \geq n$

\end{center}
We know that $f(n) \geq cn$, $\forall n \geq 0$. Thus $f(n) = \Omega(n)$.
\end{proof}
\pagebreak
\subsection{Prove that $f(n) = \Omega (n^2)$}

\begin{proof}
To prove $f(n) = \Omega(n^2)$ we must show that there exists a \textbf{positive} value, $n_{0}$, and a positive constant, $c$, such that:
\begin{center}
$ 3n^2 + 10n + 729 \geq cn $,  $\forall n > n_{0}$
\end{center}
If we can show that such an $n_{0}$ and $c$ exist, that is the definition of $\Omega$ Notation.

Let $n_{0} = 0$ and $c = 1$. We know:
\begin{center}
$3n^2 \geq n^2$, $\forall n $

$3n^2 + 10n \geq n^2$, $\forall n \geq 0$ and

$3n^2 + 10n + 729 \geq n^2$

\end{center}
So, when $n \geq 0$
\begin{center}
$3n^2 + 10n + 729 \geq 3n^2 + 10n \geq n^2$
\end{center}
Thus we know,
\begin{center}
$3n^2 + 10n + 729 \geq n^2$
\end{center}
$f(n) \geq cn^2$, $\forall n \geq n_{0} = 0$. Thus $f(n) = \Omega(n^2)$.
\end{proof}
\pagebreak

\subsection{Best Bounds for Each}
For both the Big O and $\Omega$, I believe the best bounds are the ones that bound around the function the tightest. So while $f(n) = O(n^3)$ this is not as optimal as $f(n) = O(n^2)$ since the latter forms a tighter bound on $f(n)$. Similarly I believe $\Omega(n)$ is better than $\Omega(n^2)$ for the same reason.
\end{document}
%%%%%%%%%%%%%%%%%%%%%%%%%%%%%%%%%%%%%%%%%%%%%%%%%%%%%%%%%%%%%%%%%%%%%%

