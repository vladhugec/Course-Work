%%%%%%%%%%%%%%%%%%%%%%%
% Comp 160, Fall 2019
% Homework 1
% Author: Vladimir Hugec
%%%%%%%%%%%%%%%%%%%%%%%

% This portion of the LaTeX document are configuration 
% You can see it as all the #includes in C++
\documentclass[12pt]{article}

\usepackage{epsfig}
\usepackage{amsmath}
\usepackage{amsthm}
\usepackage{listings}
\usepackage{graphicx}

\newtheorem{lemma}{Lemma}
\newtheorem{theorem}{Theorem}

\usepackage{titlesec}
\titleformat{\section}
{\normalfont\Large\bfseries}{Question~\thesection:}{1em}{}

\newlength{\toppush}
\setlength{\toppush}{2\headheight}
\addtolength{\toppush}{\headsep}

\def\subjnum{Comp 160}
\def\subjname{Introduction to Algorithms}

\def\doheading#1#2#3{\vfill\eject\vspace*{-\toppush}%
  \vbox{\hbox to\textwidth{{\bf} \subjnum: \subjname \hfil Vladimir Hugec}%
    \hbox to\textwidth{{\bf} Tufts University, Fall 2019 \hfil#3\strut}%
    \hrule}}


\newcommand{\htitle}[1]{\vspace*{1.25ex plus 1ex minus 0ex}%
\begin{center}
{\large\bf #1}
\end{center}} 


%%%%%%%%%%%%%%%%%%%%%%%%%%%%%%%%%%%%%%%%%%%%%%%%%%%%%%%%%%%%%%%%%%%
% BEGIN DOCUMENT
%%%%%%%%%%%%%%%%%%%%%%%%%%%%%%%%%%%%%%%%%%%%%%%%%%%%%%%%%%%%%%%%%%%
\begin{document}
\doheading{2}{title}{Homework 1}

\section{Self introduction}
\subsection{Photo}

\begin{figure} [h]
\begin{center}
\caption{This is a picture of me, Vladimir Hugec}

\end{center}
\end{figure}

\subsection{Hobbies}

\begin{enumerate}
	\item Tennis
	\item Computers
	\item Soccer
\end{enumerate}

\pagebreak

\section{Previous Knowledge}

\subsection{Known Topics}
	\begin{enumerate}
		\item InsertionSort, MergeSort
		\item Dynamic Programming
		\item BFS, DFS
	\end{enumerate}

\subsection{Familiar Topics}
	\begin{enumerate}
		\item big O and Omega Notation
		\item Deterministic selection (median-finding)
		\item Hashing
		\item BST and relationship to QuickSort
		\item NP-hardness
		\item Reductions
		\item Approximation
	\end{enumerate}

\subsection{Unknown Topics}
	\begin{enumerate}
		\item Recurrences by trees and substitution
		\item Master method
		\item Sorting lower bound
		\item CountingSort and RadixSort
		\item IRV and QuickSort Analysis
		\item Randomized Selection
		\item Augmented trees
		\item Red-Black trees
		\item Amortization
		\item Topological sort and SCC
		\item Kruskal’s algorithm
		\item Prim’s algorithm
		\item SSSP
	\end{enumerate}

\pagebreak

\section{Mock Exercise}

\begin{lemma}
Since we know the order in which the cities are visited, and we know the population of each of those cities, we can simply write a program that essentially models the first calendar year and returns the city with the highest population completed in that year.
\end{lemma}

\begin{proof}

We have a list of cities and their populations

\begin{enumerate}
	\item C1:P1
	\item C2:P2
	\item C3:P3
	\item .......
	\item Cn:Pn
\end{enumerate}

We start counting on a hypothetical Jan 1st or day 1/365. For each day P1/2 until P1 = 0. Then we move to P2 and repeat P2/2 until P2 = 0. Once the day is 365/365 we stop. If the city that we were currently counting, Ci, has Pi = 0 then include it in the cities completed. Then we are left with a list of cities completed in the first year and we sort by the largest Pn and return the first element.

\end{proof}

\pagebreak

\section{Fun Challenges}

\subsection{Formula}

$$ \sqrt{\alpha + \Theta - \frac{x}{2} + y^2} $$

\subsection{Table}

\begin{tabular}{|c|c|c|c|c|}
\hline
% after \\ : \hline or \cline{col1-col2} \cline{col3-col4} ...
   Monday & Tuesday & Wednesday & Thursday & Friday \\
   1:30 Tennis & 1:30 Comp170 & & 1:30 Comp170  \\
   & 4:30 170 Rec. & 5:30 160 Recitation & 6:00 Comp160  \\ 
   & 6:00 Comp160 &   &\\ 
\hline
\end{tabular}

\subsection{Figure}

\begin{figure}[h]
\begin{center}
\includegraphics[width=2in]{graph}
\caption{this is a graph of $y = x^2$}
\label{ }
\end{center}
\end{figure}

\subsection{Command}

The most 'fun' command I've come across is the Coffee stains package by Hanno Rein. It draws a coffee stain on the document like the one on the top of this page!

\cofeAm{1}{1.0}{0}{5.5cm}{3cm}



\end{document}
%%%%%%%%%%%%%%%%%%%%%%%%%%%%%%%%%%%%%%%%%%%%%%%%%%%%%%%%%%%%%%%%%%%%%%

