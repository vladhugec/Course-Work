%%%%%%%%%%%%%%%%%%%%%%%
% Comp 160, Fall 2019
% Homework 12
% Author: Vladimir Hugec
%%%%%%%%%%%%%%%%%%%%%%%

% This portion of the LaTeX document are configuration 
% You can see it as all the #includes in C++
\documentclass[12pt]{article}

\usepackage{epsfig}
\usepackage{amsmath}
\usepackage{amsthm}
\usepackage{listings}
\usepackage{graphicx}
\usepackage{tikz}
\usepackage{colortbl}
\usepackage{multirow}

\newtheorem{lemma}{Lemma}
\newtheorem{theorem}{Theorem}

\usepackage{titlesec}
\titleformat{\section}
{\normalfont\Large\bfseries}{Question~\thesection:}{1em}{}

\newlength{\toppush}
\setlength{\toppush}{2\headheight}
\addtolength{\toppush}{\headsep}

\def\subjnum{Comp 160}
\def\subjname{Introduction to Algorithms}

\def\doheading#1#2#3{\vfill\eject\vspace*{-\toppush}%
  \vbox{\hbox to\textwidth{{\bf} \subjnum: \subjname \hfil Vladimir Hugec}%
    \hbox to\textwidth{{\bf} Tufts University, Fall 2019 \hfil#3\strut}%
    \hrule}}


\newcommand{\htitle}[1]{\vspace*{1.25ex plus 1ex minus 0ex}%
\begin{center}
{\large\bf #1}
\end{center}} 


%%%%%%%%%%%%%%%%%%%%%%%%%%%%%%%%%%%%%%%%%%%%%%%%%%%%%%%%%%%%%%%%%%%
% BEGIN DOCUMENT
%%%%%%%%%%%%%%%%%%%%%%%%%%%%%%%%%%%%%%%%%%%%%%%%%%%%%%%%%%%%%%%%%%%
\begin{document}
\doheading{2}{title}{Homework 13 Part 2}

\section{Problem 3}

Stable: Bubble sort, Insertion sort, Merge sort, Counting sort, Radix Sort $\newline$
Not Stable: Quick sort, Heap sort, Selection sort $\newline$
Can be made stable: All non-stable can be made into stable by taking into account the positions of the elements prior to the sort $\newline$

\pagebreak

\section{Problem 4}

Yes, $SELECTION$ with groups of 3 does find the $k$-th smallest element, however it will not run in linear time since we don't reduce the subproblems as efficiently as with groups of 5. With groups of 3 we are still left with subproblems of size $n$ whereas with groups of 5 and larger the problem get reduced to size of less than $n$.

\pagebreak

\section{Problem 7}

QUICKSORT: Depending on implementation, chooses either the first or last element of array as pivot.$\newline$
SELECTION: Just as QUICKSORT, chooses either the first or last element as pivot.$\newline$
RANDOMIZED SELECTION: The pivot is chosen at random uniformly using a random number generator from (start,end)$\newline$

\pagebreak

\section{Problem 21}

For a vertex to be a cut vertex as a function of its discovery time, for two nodes, $u, v$, where $u$ is a parent and $v$ is a child. Let $d[v]$ be the discovery time for a node $v$. Let $L[u]$ be the lowest node that can be reached from any vertex. Then if $L[v] \geq d[u]$ then  $u$ is a cut vertex.









 

\end{document}
%%%%%%%%%%%%%%%%%%%%%%%%%%%%%%%%%%%%%%%%%%%%%%%%%%%%%%%%%%%%%%%%%%%%%%

