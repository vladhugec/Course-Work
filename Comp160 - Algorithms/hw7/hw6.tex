%%%%%%%%%%%%%%%%%%%%%%%
% Comp 160, Fall 2019
% Homework 7
% Author: Vladimir Hugec
%%%%%%%%%%%%%%%%%%%%%%%

% This portion of the LaTeX document are configuration 
% You can see it as all the #includes in C++
\documentclass[12pt]{article}

\usepackage{epsfig}
\usepackage{amsmath}
\usepackage{amsthm}
\usepackage{listings}
\usepackage{graphicx}

\newtheorem{lemma}{Lemma}
\newtheorem{theorem}{Theorem}

\usepackage{titlesec}
\titleformat{\section}
{\normalfont\Large\bfseries}{Question~\thesection:}{1em}{}

\newlength{\toppush}
\setlength{\toppush}{2\headheight}
\addtolength{\toppush}{\headsep}

\def\subjnum{Comp 160}
\def\subjname{Introduction to Algorithms}

\def\doheading#1#2#3{\vfill\eject\vspace*{-\toppush}%
  \vbox{\hbox to\textwidth{{\bf} \subjnum: \subjname \hfil Vladimir Hugec}%
    \hbox to\textwidth{{\bf} Tufts University, Fall 2019 \hfil#3\strut}%
    \hrule}}


\newcommand{\htitle}[1]{\vspace*{1.25ex plus 1ex minus 0ex}%
\begin{center}
{\large\bf #1}
\end{center}} 


%%%%%%%%%%%%%%%%%%%%%%%%%%%%%%%%%%%%%%%%%%%%%%%%%%%%%%%%%%%%%%%%%%%
% BEGIN DOCUMENT
%%%%%%%%%%%%%%%%%%%%%%%%%%%%%%%%%%%%%%%%%%%%%%%%%%%%%%%%%%%%%%%%%%%
\begin{document}
\doheading{2}{title}{Homework 7}

\section{}

\subsection{A}

The worst-case runtime of a single $MULTIPOP$ operation is $O(n)$ since in a single operation, the most you can pop is the size of the stack.

\subsection{B}

A sequence of $n$ operations in any combination of $PUSH$, $POP$, and $MULTIPOP$ should take $O(n)$ time since for an element $PUSH$'ed to the stack it can be popped at most once. And so, the number of times that a $POP$ operation can be executed is at most the number of times that an element has been $PUSH$'ed, which is at most $n$, so $O(n)$.

\subsection{C}

\textbf{What virtual cost would you give to each operation?}

$PUSH \rightarrow 2$

$POP \rightarrow 0 $

$MULTIPOP \rightarrow 0$

$\newline$
\textbf{What is the real cost of each operation?}

$PUSH \rightarrow O(1)$ 

$POP \rightarrow O(1)$

$MULTIPOP \rightarrow O(min(k,n))$

$k$ is the number of elements to be $POP$'ed

$n$ is the number of elements in the stack.

\subsection{D}

When we push to the stack we pre-pay a cost of $1$ coin, thus each element in the stack has been half paid for. So when we pop an element, we also remove the pre-paid cost and use it to pay the $POP$ cost, so we charge the operation nothing but use the pre-paid cost for payment. This is the same for a $MULTIPOP$ operation, since that is just a series of $POP$ operations. Therefore we always have enough stored up in pre-paid amounts to pay for the $POP$ and $MULTIPOP$ operations. And since the stack is always non-negative we also ensure that the amount stored up in credit is always non-negative as well so we will never run out of coins.

\subsection{E}

Since we know:

$\bullet $ That for any sequence of $n$ $PUSH$, $POP$, and $MULTIPOP$ operations, amortized cost upper bounds the total cost by definition.

$\bullet$ And the total credit stored in the stack is the difference between the amortized cost and the total cost.

$\bullet$ Therefore the total credit is always nonnegative.

$\bullet$ We also know that for any $POP$ operation, there must have been a previous $PUSH$ operation.

$\bullet$ Therefore we know that since the amortized cost is $O(n)$ then the total cost must also be $O(n)$

\subsection{F}

Since a $MULTIPUSH$ operation is simply a series of $PUSH$ operations executed $k$ times on an object. The amortized cost of the $MULTIPUSH$ operation would be $2k$. This is not different, however, from individually pushing an object $k$ times and so the total cost would be dependent on the number of elements in the stack and not the method of pushing. So $POP$ and $MULTIPOP$ have unchanged amortized costs and the total amortized cost would still be $O(n)$ and so too the total actual cost.

\pagebreak

\section{}

\subsection{A}

I would use an ArrayList.

\subsection{B}

To insert, I would simply insert at the end of the array if there is space left. Therefore insertion would take $O(1)$ time. If there is no space left, create a new array of size $2n$ and copy over the $n$ elements, $O(n)$ time. So for the first $n$ elements insertion is constant time.

\subsection{C}

To delete the largest $\lceil \frac{n}{2} \rceil$ elements, since to achieve constant insertion the array would need to be unsorted, I would use the median method to split the array into elements larger than the median and smaller than the median. That way we are left with a half of the array that is strictly larger than the other half. Deleting that half would be done in constant $O(1)$ time and partitioning by relation to the median would be done in $O(n)$ time.

\subsection{D}

Since we probably don't want to measure exact runtime, an individual operation can take at the worst case $O(n)$ time, I'm going to assess runtime in the average case where there are a sequence of operations being performed.

\subsection{E}

The potential function $\Phi$ would be defined as the number of elements present in the Array. The potential of the empty Array, $D_{0}$ would be set to $0$. So $\Phi(D_{0}) = 0$.

\pagebreak

\subsection{F}

From the TextBook we know $\hat{c_{i}} = c_{i} + \Phi(D_{i}) - \Phi(D_{i-1})$.

For the $INSERT$ operation, we know that for an Array containing $n$ elements, after the operation it would contain $n+1$ elements. Therefore we know $\Phi(D_{i}) - \Phi(D_{i-1}) = (n+1) - n = 1$. So $\hat{c_{i}} = c_{i} + 1$. Since $c_{i}$ is the actual cost of the insert operation. Assuming the insertion is in an array containing $n-1$ elements, $c_{i} = 1$. Therefore for the $INSERT$ operation we have $\hat{c_{i}} = 2$, so $\hat{c_{i}}$ for $INSERT$ is constant.

For the $DELETE-HALF$ operation, we know that for an Array containing $n$ elements, after the operation it would contain $\lfloor \frac{n}{2} \rfloor$ elements. Therefore we know $\Phi(D_{i}) - \Phi(D_{i-1}) = \frac{n}{2} - n = \frac{-n}{2}$. So $\hat{c_{i}} = c_{i} - \frac{n}{2}$. Since $c_{i}$ is the actual cost of the $DELETE-HALF$ operation, $c_{i} = \frac{n}{2}$ and so we have $\hat{c_{i}} = 0$.

\subsection{G}



$\indent\bullet$ Since we know by our definition of $\Phi$ that $\Phi(D_{i}) \geq \Phi(D_{0})$ (because the number of elements in the array will never be negative)

$\bullet$ This means that the total amortized cost is an upper bound on the total cost.

$\bullet$ So for a sequence of $m$ operations, we have $O(m)$




\end{document}
%%%%%%%%%%%%%%%%%%%%%%%%%%%%%%%%%%%%%%%%%%%%%%%%%%%%%%%%%%%%%%%%%%%%%%

